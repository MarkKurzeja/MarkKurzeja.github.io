%%%%%%%%%%%%%%%%%%%%%%%%%%%%%%%%%%%%%%%%%%%%%%%%%%%%%%%%%%%%%%%%%%%%%%%%%%%%%%%%
%                                                                              %
% Purpose:       Dense Three Column Summary                                    %
%                                                                              %
% Author:        Mark Kurzeja                                                  %
% Contact:       mtkurzej@umich.edu                                            %
% Client:        Mark Kurzeja                                                  %
%                                                                              %
% Code creation: 2015-10-10                                                    %
%                                                                              %
% Comment:       Create three column dense info sheets for the presentation o  %
%                f concise information or to quickly condense a topic or a     %
%                class into a handout                                          %
%                                                                              %
% License:       MIT Open Source License                                       %
%                                                                              %
%%%%%%%%%%%%%%%%%%%%%%%%%%%%%%%%%%%%%%%%%%%%%%%%%%%%%%%%%%%%%%%%%%%%%%%%%%%%%%%%

\documentclass[landscape]{article}

% Opening
\title{}
\author{}

% Packages
\usepackage{multicol}
% Control the geometry of the page with 1/2 inch margins
\usepackage[width = 10.5in, height = 8in]{geometry}
\usepackage{xspace}
\usepackage{nicefrac}
\usepackage{amsmath}
\usepackage{lipsum}
\usepackage{amsfonts}
\usepackage{graphicx}
\usepackage{wasysym} % Fullmoon
\usepackage{float}
\usepackage{todonotes}
\usepackage{booktabs}
\usepackage{amssymb}
\usepackage{makecell} % for \makecell{.... \\ .....} for multiline cells
\usepackage{changepage}   % for the adjustwidth environmen
\usepackage{siunitx} % For the num operator in producing scientific notation
\usepackage{multicol}

% Fonts selection
\usepackage[]{kpfonts}
%\usepackage[regular, default]{sourcesanspro} % Very close to Myriad - favorite sans serif % option light,scale
%\usepackage{alegreya}
%\usepackage[sfdefault]{universalis}
%\usepackage{ebgaramond}
%\usepackage{accanthis}
%\usepackage{libertine}
%\renewcommand{\familydefault}{\sfdefault}

% ==============================================================================
% Listings Declaration
% ==============================================================================

\usepackage{listings} 
%\lstset{language=R} 
\usepackage{color}
\definecolor{mygreen}{rgb}{0,0.6,0}
\definecolor{mygray}{rgb}{0.5,0.5,0.5}
\definecolor{mymauve}{rgb}{0.58,0,0.82}

\lstset{ %
	backgroundcolor=\color{white},   % choose the background color; you must add \usepackage{color} or \usepackage{xcolor}
	basicstyle=\scriptsize,        % the size of the fonts that are used for the code
	breakatwhitespace=false,         % sets if automatic breaks should only happen at whitespace
	breaklines=true,                 % sets automatic line breaking
	captionpos=b,                    % sets the caption-position to bottom
	commentstyle=\color{mygreen},    % comment style
	deletekeywords={...},            % if you want to delete keywords from the given language
	escapeinside={\%*}{*)},          % if you want to add LaTeX within your code
	extendedchars=true,              % lets you use non-ASCII characters; for 8-bits encodings only, does not work with UTF-8
	frame=single,	                   % adds a frame around the code
	keepspaces=true,                 % keeps spaces in text, useful for keeping indentation of code (possibly needs columns=flexible)
	keywordstyle=\color{blue},       % keyword style
	language=R,                 % the language of the code
	otherkeywords={*,...},            % if you want to add more keywords to the set
	numbers=left,                    % where to put the line-numbers; possible values are (none, left, right)
	numbersep=5pt,                   % how far the line-numbers are from the code
	numberstyle=\tiny\color{mygray}, % the style that is used for the line-numbers
	rulecolor=\color{black},         % if not set, the frame-color may be changed on line-breaks within not-black text (e.g. comments (green here))
	showspaces=false,                % show spaces everywhere adding particular underscores; it overrides 'showstringspaces'
	showstringspaces=false,          % underline spaces within strings only
	showtabs=false,                  % show tabs within strings adding particular underscores
	stepnumber=1,                    % the step between two line-numbers. If it's 1, each line will be numbered
	stringstyle=\color{mymauve},     % string literal style
	tabsize=2,	                   % sets default tabsize to 2 spaces
	title=\lstname                   % show the filename of files included with \lstinputlisting; also try caption instead of title
}

% ==============================================================================
% Commands Declaration
% ==============================================================================

% For collapsing the lists to a single condensed itemized list
\usepackage{enumitem}
\setlist{nosep} % or \setlist{noitemsep} to leave space around whole list

% Specific functions that remove the issues with the spacing around an align enviroment
\usepackage{etoolbox}
\newcommand{\zerodisplayskips}{
	\setlength{\abovedisplayskip}{2pt}
	\setlength{\belowdisplayskip}{2pt}
	\setlength{\abovedisplayshortskip}{2pt}
	\setlength{\belowdisplayshortskip}{2pt}
}
\appto{\normalsize}{\zerodisplayskips}
\appto{\small}{\zerodisplayskips}
\appto{\footnotesize}{\zerodisplayskips}

% A simple figure environment for inserting a picture into the column
% Arguments:
% 1: Name of the picture file
% 2: Caption for the picture
\newcommand{\qpics}[2]{
	\begin{figure}[H]
		\centering
		\includegraphics[width=0.6\linewidth]{./#1}
		\label{fig:#1}
	\end{figure}
}

% Command for creating a very specific spaced hrule that wont interfere with the 
% text around it
\newcommand{\myline}{\vspace{4pt}\hrule  \vspace{4pt}}

% Command for breaking the current line of thought into a new column, and placing
% a new line above the next column
\newcommand{\mynewcolumn}{\columnbreak \myline}

% Home-made inline bullet points for compact lists
\newcommand{\mydot}{\ensuremath{\bullet}\xspace} %\ensuremath{\cdot}\xspace}

% Command for inserting a legal blob!
\newcommand{\legalblob}{\ensuremath{ \left[ \newmoon \right] } \xspace}

% Command that adds a space after the use of an item command in a description 
% environment
% Description Environment Item Helper Commands
\newcommand{\im}[1]{\item[#1] \xspace}
\newcommand{\imp}[1]{\item[(#1)] \xspace}

% Auto-commas for long nominal and dollar amounts
\RequirePackage{siunitx}
\newcommand{\commasep}[1]{\num[group-separator={,}]{#1}}
\newcommand{\money}[1]{\$\commasep{#1}}

% For mathematical expectations enclosure
\newcommand{\expt}[1]{\ensuremath{\mathbb{E}\left[ #1\right] }\xspace}

% ==============================================================================
% The main topic command which is the workhorse of this document
% Command that allows one to generate topics quickly
% Arguments:
% 1: Header for this particular topic
% 2: Body of the topic
% ==============================================================================

\definecolor{harvardcrimson}{HTML}{A41034} % For the main headings
\definecolor{harvardblue}{HTML}{0D667F} % For the sub headings
%\definecolor{brewerred}{HTML}{E41A1C}
%\definecolor{brewerblue}{HTML}{377EB8}
%\definecolor{brewergreen}{HTML}{4DAF4A}
%\definecolor{harvardgrey}{HTML}{B6B6B6}
%\definecolor{harvarddarkgrey}{HTML}{808285}

% ==============================================================================
% How to use this document:
% Use \mynewcolumn to start a new column
% Use \topic to start another mini topic -> main command
% Use \ctopic to clear a topic from existence without commenting it
% Use \mydot to get a bullet figure for quicklists (lists for which you don't 
%      have a line break)
% Use \myline to get an hrule in a place that you wish
% ==============================================================================

% Main command for creating a topic which is the main unit of parsing 
\newenvironment{topic}[1]{
	\noindent \textbf{\textsc{\color{harvardcrimson}{#1}}}
	\noindent \hspace{-3.5pt}
}{
	\myline
}

% A method for clearing out a topic command without commenting it out
\newenvironment{ctopic}[1]{
}{
}

% For creating super neat indented sub paragraphs
\newenvironment{tellme}[1]{
	\noindent \textbf{\textit{\color{harvardblue}{#1}}}
	\begin{adjustwidth}{9pt}{}
	}{
	\end{adjustwidth}
}

% For creating item enviroments that are flush left with the margin
\newenvironment{compactitem}{
	\begin{itemize}[leftmargin=*,labelsep=5pt]
	}{
	\end{itemize}
}
% For creating enumeration enviroments that are flush left with the margin
\newenvironment{compactenum}{
	\begin{enumerate}[leftmargin=*,labelsep=5pt]
	}{
	\end{enumerate}
}

% For creating description enviroments that have a smaller margin
\newenvironment{compactdesc}{
	\begin{description}[leftmargin=1em,labelsep=0.7em, font=\normalfont\itshape]
	}{
	\end{description}
}

%%%%%%%%%%%%%%%%%%%%%%%%%%%%%%%%%%%%%%%%%%%%%%%%%%%%%%%%%%%%%%%%%%%%%%%%%%%%%%%%
%                                                                              %
%                                Begin Document                                %
%                                                                              %
%%%%%%%%%%%%%%%%%%%%%%%%%%%%%%%%%%%%%%%%%%%%%%%%%%%%%%%%%%%%%%%%%%%%%%%%%%%%%%%%

\begin{document}

	\footnotesize

	% Begin the multicols environment
	\begin{multicols*}{3}
	% Make the title align with the rest of the columns
	\hfill
	\vspace{-1\baselineskip}
	\hfill
	
	% Making the Title
	\myline
	\vspace{-0.2cm}
	\begin{center}
		\LARGE \textsc{First Exam Study Guide} 
	\end{center}
	\vspace{-0.2cm}
	\myline 
	
	% Template for a new topic
		
	\begin{topic}{Introduction}
		This summary is not meant to be comprehensive. By using this document, you acknowledge you have read the disclaimer.
		\begin{compactdesc}
			\item[Intrepretations] Look back at Exam One and Exam Two for the interpretations of things like p-values and other vocabulary. Remember to be concise when presenting these on an exam - sometimes saying more than necessary can hurt more than it helps.
			\item[The Formula Card] Knowing how to quickly utilize the formula card is helpful for many of the problems you may encounter. The notation on the formula card is correct, and a good habit is to always check the formula card whenever you feel you have hit a dead-end. Often, a formula or hint on the card can provide the next step to finish the problem. 
			\item[Name that scenerio] Being able to identify which test you are looking at quickly and accurately is invaluable. The assumptions, interpretations, distributions, and conclusions all hinge on selecting the correct test, so be sure to study name-that-scenario questions
			\item[Broad Concepts] The final will test your understanding of broad concepts in this course. Make sure you have a grasp of the big picture as well as the methodology behind each test. 
			\item[Remember] To get a good nights sleep and practice taking an exam towards the later part of the day. This is a later exam than most. Be sure to prepare yourself by taking a practice exam around the time of our actual exam. 
		\end{compactdesc}
	\end{topic}
	
	\begin{topic}{General Probability Questions}
		Probability is one of the more difficult units in the class because it sometimes is difficult to decipher what the question is asking. Remember a few simple rules before you start any probability question to start off on the right foot:
		\begin{compactenum}
			\item Most problems involve using one or at most two formulas on the formula card. Begin by defining terms. For example, if we are looking at the probability of it raining vs the probability of you smiling, then let $ A =$ ``It is raining'' and let $ B =  $ ``You are Smiling''. Then we can use the formulas on our formula card directly.
			\item When you see something that looks \textit{like} coin flips then you are probably dealing with a binomial problem. For instance, if we are trying to see what the probability of 10 out of our 15 neighbors love chocolate, and we know that the probability of loving chocolate is 90\%, then this is probably a binomial problem. For this we need:
			\begin{compactitem}
				\item It has a yes/no style question
				\item It has a fixed number of ``flips'' 
			\end{compactitem}
		\end{compactenum}
	\end{topic}
	
	\begin{topic}{Mutual Exclusivity and Independence}
		These two are \textit{not} the same thing! The exams generally test this
		\begin{compactdesc}
			\item[Mutual Exclusivity] This says that the two events cannot happen at the same time. If so $ P(A \text{ and } B) = 0 $. Example: If we flip a coin, we cannot get a head and a tail in the same toss 
			\item[Independence] This says that knowing something about $ A $ will tell us nothing about $ B $. The formula on the formula card is: $ P(A | B) = P(A) $, but an easier formula for these problems is to check does this hold: $ P(A \text{ and }  B) = P(A) \times P(B) $. This formula is not on your formula card but always comes in handy. Example: If I flip a head on turn one and the probability of flipping a head is 50\%, the probability that I flip a head on turn two is still 50\% on turn two!
		\end{compactdesc}
		Something \textit{cannot} be both mutually exclusive and independent! 
	\end{topic} 	
	\begin{topic}{Probability Example}
		Let's imagine that we saw the following on the exam: 
		\begin{quote}
			The probability of it raining is 30\%. The probability of you not smiling is 20\%. The probability of it raining or you smiling is 50\%.
		\end{quote}
		\begin{compactdesc}
			\item[Problem Setup] First thing's first: always set up the problem by putting labels to things! We will let $ A = \text{Probability of it Raining} $ and $ B = \text{Probability of you smiling} $
			\item[Writing out the problem in Symbols] This step makes things a lot easier since we can now use our formula card
			\begin{align*}
			P(A) &= 30\% \text{ Given in the problem}\\
			P(\text{Not }B) &= 20\% \text{ We have the probabilty of not smiling}\\
			P(A \text{ or } B) &= 50\% 
			\end{align*} 
			\item[Solving for things we want] Let's find the probability of smiling
			\begin{align*}
			P(B) = 1 - P(\text{Not }B) = 1 - 20\% = 80\%
			\end{align*}  
			\item[Are Smiling and Raining Mutually Exclusive?] This is where writing out the problem in symbols helps out a lot because we can use what we have from the formula card:
			\begin{align*}
			\underbrace{P(A \text{ or } B)}_{50\%} &= \underbrace{P(A)}_{30\%} + \underbrace{P(B)}_{80\%} - \underbrace{P(A\text{ and }B)}_{???}
			\intertext{and solving out the algebra gives us:}
			P(A\text{ and }B) &= P(A \text{ or }B) - P(A) - P(B)\\
			&= 60\%
			\end{align*}
			Because $ P(A\text{ and }B) $ is not equal to zero, we can say that Smiling and Raining are not mutually exclusive!
			\item[Are Smiling and Raining Independent?] We can use the fact that Smiling and Raining are Independent only if $ P(A) \times P(B) = P(A \text{ and }B) $. Does this hold? We will use the answer from the last problem to help us. 
			\begin{align*}
			\underbrace{P(A)}_{30\%} \times \underbrace{P(B)}_{80\%} &\overset{?}{=} \underbrace{P(A\text{ and } B)}_{60\%}\\
			24\% &\neq 60\%
			\end{align*} 
			and thus we can see that $ A $ and $ B $ (Raining and Smiling) are not independent because the formula is not equal on both sides. 
		\end{compactdesc}
	\end{topic}
	
	\begin{topic}{Binomial Example}
		Lets go over how a binomial problem shows up on the exam. 
		\begin{quote}
			You are watching CSPAN because you are into politics and you find out (to your surprise) that only 3\% of Americans watch CSPAN. You're wondering what the probability that at least one of your neighbors watches CSPAN so you can watch it with them. Assume you have 25 neighbors.
		\end{quote}
		\begin{compactdesc}
			\item[Why is this a Binomial Problem?]
			\begin{compactenum}
				\item Binary Choice - Each neighbor either watches CSPAN or does not. These are the only two choices
				\item Fixed $ p $ - We assume that the probability of one neighbor watching CSPAN is the same as any other neighbor
				\item Fixed $ n $ - We have a total of 25 neighbors and this doesn't change
			\end{compactenum} 
			\item[What is the probability that if you have one neighbor they watch CSPAN?] This is given in the problem. This is 3\%
			\item[What is the probability that exactly 4 people watch CSPAN] We are looking for $ P(X = 4) $ i.e. four of our neighbors watch CSPAN. $ X $ is distributed binomial with $ n = 25 $ and $ p = 0.03 $ which can also be written $ X \sim Binomial(n = 25, p = 0.03) $. We can use the formula card to see that $ P(X = 4) = \binom{25}{4} 4^{0.03} 21^{0.97} = 0.00540$
			\item[What is the probabilty that at least one neighbor watches CSPAN?] We are looking for $ P(X \geq 1) = P(X = 1) + P(X = 2) + \cdots + P(X = 25)$. Probabilities sum to one, so we can rewrite this as: $ P(X\geq 1) = 1 - P(X = 0) $. Lets find the probability that $ P(X = 0) $. Using the Binomial formula:
			\begin{align*}
			P(X\geq 1) &= 1 - P(X = 0)\\
			P(X\geq 1) &= 1 - \binom{25}{0} 0^{0.03} 25^{0.97}\\
			P(X\geq 1) &= 1 - 0.4669747\\
			P(X\geq 1) &= 0.5330253\\
			\end{align*}  
			\item[What is the expected number of neighbors that watch CSPAN?] The ``expected value'' of a binomial is another way of asking for the mean. The mean of the binomial is also given on the formula card: $ E[X] = np = 25\times 0.03 = 0.75 $ people on average. 
		\end{compactdesc}
	\end{topic}
	
	\begin{topic}{One Proportion Test Assumptions}
		Let's go through a One Proportion problem in all of the ways that it can play out:
		\begin{quote}
			Imagine that your friend says that only 70\% of people in Ann Arbor love dogs. You think this is too low so you decide to run a test
		\end{quote}
		Statement: $ H_0: p = 0.7, H_a: p > 0.7 $ where $ p $ is the population proportion of UofM students who love dogs. Lets look at two different scenerios:
	
		\begin{compactitem}
			\item Imagine that you poll 100 people and 90 of them say they love dogs. We first check assumptions:
			\begin{center}
				\begin{tabular}{cccc}
					State & Check & Values & $ \geq 10 $ \\
					\midrule
					$ np \geq 10 $    &    $ np_0 \geq 10 $   &    $ 100 \times 0.7 $    &      True      \\
					$ n(1-p) \geq 10 $    &    $ n(1-p_0) \geq 10 $   &     $ 100 \times (1-0.7) $   & True
				\end{tabular}
			\end{center}
			and since our assumptions are satisfied, we can proceed with the Large Sample Z Test:
			\begin{align*}
			X &\sim Binomial(n = 100, p = 0.7) \text{ (in reality)}\\
			X &\approx N(\mu = 0.7, \sigma = \sqrt{\nicefrac{1}{100} (0.7)(1-0.7)}) \text{ (Approximation)}\\
			Z &\sim N(0,1) \text{ (Distribution of Test Statistic)}\\
			Z &= \frac{\hat{p} - p_0}{\sqrt{\frac{p_0(1- p_0)}{n}}} = \frac{0.9 - 0.7}{\sqrt{\frac{0.7(1-0.7)}{100}}} = 95.23\\
			P(Z > 95.23) &= \text{ by our z-table } \approx 0 
			\end{align*}
			\item Now, imagine instead that you only decided to poll 20 people and 18 say they love dogs:
			\begin{center}
				\begin{tabular}{cccc}
					State & Check & Values & $ \geq 10 $ \\
					\midrule
					$ np \geq 10 $    &    $ np_0 \geq 10 $   &    $ 20 \times 0.7 $    &      True      \\
					$ n(1-p) \geq 10 $    &    $ n(1-p_0) \geq 10 $   &     $ 20 \times (1-0.7) $   & False
				\end{tabular}
			\end{center}
			Our assumption is not satisfied. So we need to use the binomial distribution to compute the p-value. 
			\begin{align*}
				X &\sim Binomial(n = 20, p = 0.7)\\
				P(X \geq 18) &= P(X = 18) + P(X = 19) + P(X = 20)\\
				\intertext{which by the binomial formula is:}
				P(X \geq 18) &= \binom{20}{18} 18^{0.7} 2^{0.3} + \binom{20}{19} 19{0.7} 1^{0.3} + \binom{20}{20} 20^{0.7} 0^{0.3}\\
				P(X \geq 18) &= 0.03548313
			\end{align*}
			and this will now be our p-value.
		\end{compactitem}
	\end{topic}
	
	\begin{topic}{}
		
	\end{topic}
	
	
	%%%%%%%%%%%%%%%%%%%%%%%%%%%%%%%%%%%%%%%%%%%%%%%%%%%%%%%%%%%
	% Disclaimer
	%%%%%%%%%%%%%%%%%%%%%%%%%%%%%%%%%%%%%%%%%%%%%%%%%%%%%%%%%%%
	\begin{topic}{Disclaimer} \tiny
		Mark has made every attempt to ensure the accuracy and reliability of the information provided in this study material. However, the information is provided "as is" without warranty of any kind. Mark does not accept any responsibility or liability for the accuracy, content, completeness, legality, or reliability of the information contained in this study material.
		No warranties, promises and/or representations of any kind, expressed or implied, are given as to the nature, standard, accuracy, content, completeness, reliability, or otherwise of the information provided in this study material nor to the suitability or otherwise of the information to Stats 250. 
		\begin{center}
			Version: 9000.01\\
			Please email typos to \texttt{mtkurzej@umich.edu}\\
			Copyright Mark Kurzeja 2019				
%			Released under License: Attribution-NonCommerical-ShareAlike 4.0 International (CC BY-NC-SA 4.0)
		\end{center}
	
	
	\end{topic}

		

\end{multicols*}
\end{document}
